\chapter{Introduction}

This paper proposes a straightforward extension of an empirical Bayes
inferential method, LIMMA \cite{smyth2004linear}, to general asymptotically
linear estimators \cite{tsiatis2007semiparametric,van2011targeted}. This means
that more complex parameters and estimators in the context of many comparisons
can benefit of the inferential robustness that LIMMA provides. As a motivating
example, consider a previous study of mRNA expression and occupational exposure
to benzene \cite{mchale2011global}. The data consisted of around 22,000 genes
(\textit{Illumina Human Ref-8 BeadChips} platform) on 125  subjects in factories
in China. The variable of interest was occupational benzene exposure (measured
in various ways), but also information on confounding factors were also recorded
(e.g., gender, smoking status). Taking benzene exposure to be binary, the
question of interest regarded the adjusted association of each of the 20K+
expression values with benzene exposure. One could easily use the LIMMA approach
by fitting a parametric linear model with say benzene as outcome and both
exposure and confounders as predictors, and performing a multiple comparison
analysis on the estimated coefficients associated with benzene. However, one
might want to use a more nonparametric procedure, specifically one that
estimates a nonparametric estimand, where fitting of the model predicting
benzene could be done via automated, data-adaptive techniques (e.g., machine
learning). We show that utilizing LIMMA in situations such as this is possible
if asymptotically linear estimators are used, that is, where the estimator minus
the true parameter value can be approximated by an i.i.d. sum of random
variables (called the influence curve). Many complex parameters have
asymptotically linear estimators, and so with small modifications, LIMMA can be
applied to a wide variety of settings. This is particularly valuable in smaller
samples, as sampling distribution estimates for these complex estimators can be
unstable, yielding false positives, and LIMMA can ameliorate their performance
by borrowing estimates of the sampling variability across the variables of
interests (in our case, gene expressions). In this way, one can use more
data-adaptive methods to avoid the bias of arbitrary parametric assumptions
(common in bioinformatic applications), while still providing a degree of
robustness for this sometimes unstable estimators.

In the following sections, we first present in detail a data-adaptive, machine
learning-based estimator of a well-known estimand for deriving adjusted
associations. We then show how one can use the machinery of LIMMA to derive an
empirical Bayes estimate of the standard error of this estimator (and, more
generally, any asymptotically linear estimator) and finally apply the resulting
procedure to the genomic example (benzene occupational exposure) noted above.
