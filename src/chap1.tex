\chapter{Introduction}

This thesis proposes a straightforward extension of an empirical Bayes
inferential method --- the moderated statistics of Smyth, as implemented in the
popular ``limma'' software package \cite{smyth2004linear} --- for general use
with asymptotically linear parameters \cite{tsiatis2007semiparametric,
van2011targeted}. By way of this extension, estimators of complex target
parameters can benefit from the inferential robustness that such moderated
statistics provide, in the context of many comparisons. As a motivating example,
consider a previous study of miRNA expression and occupational exposure to
benzene \cite{mchale2011global}: The data consists of around $22,000$ genes
(measured via the \textit{Illumina Human Ref-8 BeadChips} platform) on $125$
subjects in factories in China. In this study, the variable of interest was
occupational exposure to benzene (measured in various ways), though information
on confounding factors was also recorded (e.g., gender, smoking status). Taking
benzene exposure to be binary, the quantity of interest can be framed as the
adjusted association of each of the roughly $22,000$ expression values with
exposure. One could easily use the approach based on moderated statistics by
fitting a parametric linear model with, say, benzene as outcome and both
exposure and confounders as predictors, performing a multiple comparison
correction on the estimated coefficients associated with benzene. However, it is
generally desirable to utilize a procedure that is less reliant on arbitrary
assumptions, specifically one estimating a nonparametric estimand, where fitting
the model of interest could be performed via automated, data-adaptive techniques
(i.e., machine learning). We show that utilizing moderated statistics in such
situations is possible if asymptotically linear parameters are used --- that is,
where the difference between the values taken by the estimator and the parameter
may be approximated by a sum of i.i.d.~random variables (i.e., the influence
curve representation). Many complex parameters have representations that are
asymptotically linear, and so with minor modifications, moderated statistics can
be applied to a wide variety of settings. This is particularly valuable in
smaller samples, as sampling distribution estimates for these complex estimators
can be unstable, yielding false positives, a problem that moderated statistics
are well-suited to ameliorate by borrowing estimates of the sampling variability
across the variables of interest (in our case, gene expression measures). In
this way, one can use data-adaptive methods to avoid the bias of arbitrary
parametric assumptions, which are common in bioinformatic applications, while
still adding a degree of robustness for these potentially unstable estimators.

In the following sections, we first detail a data-adaptive, machine
learning-based estimator of a well-known estimand for deriving adjusted
associations. We then show how the machinery of moderated statistics can be used
to derive an empirical Bayes estimate of the standard error of this estimator
--- and, generally, for any asymptotically linear estimator. Finally, we apply
the resulting procedure to the example of occupational benzene exposure
previously described.
