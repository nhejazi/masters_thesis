\usepackage{graphicx}

\usepackage{chngcntr}
\counterwithout{figure}{chapter}

\usepackage{amsmath}
\usepackage{amsfonts}
\usepackage{amssymb}
\usepackage{dsfont}
\usepackage{mathptmx}

%\usepackage{amsthm}
%\usepackage{thmtools}
%\declaretheorem[name=Theorem]{thm}
%\declaretheorem[style=definition]{example}
%\declaretheorem[style=definition]{definition}

%-----------------------------------------------------------------------------
% Special-purpose color definitions (dark enough to print OK in black and white)
\usepackage{color}

% A few colors to replace the defaults for certain link types
\definecolor{orange}{cmyk}{0,0.4,0.8,0.2}
\definecolor{darkorange}{rgb}{.71,0.21,0.01}
\definecolor{darkgreen}{rgb}{.12,.54,.11}

%-----------------------------------------------------------------------------
% The hyperref package gives us a pdf with properly built
% internal navigation ('pdf bookmarks' for the table of contents,
% internal cross-reference links, web links for URLs, etc.)
\usepackage{hyperref}

%\hypersetup{pdftex,  % needed for pdflatex
%  breaklinks=true,  % so long urls are correctly broken across lines
%  colorlinks=true,
%  urlcolor=blue,
%  linkcolor=darkorange,
%  citecolor=darkgreen,
%  }

\hypersetup{pdftex,  % needed for pdflatex
  breaklinks=true,  % so long urls are correctly broken across lines
  colorlinks=true,
  urlcolor=black,
  linkcolor=black,
  citecolor=black,
}

\usepackage{url}

%% Define a new 'leo' style for the package that will use a smaller font.
\makeatletter
\def\url@leostyle{%
  \@ifundefined{selectfont}{\def\UrlFont{\sf}}{\def\UrlFont{\small\ttfamily}}}
\makeatother
%% Now actually use the newly defined style.
\urlstyle{leo}

% enables straight single quote
\makeatletter
\let \@sverbatim \@verbatim
\def \@verbatim {\@sverbatim \verbatimplus}
{\catcode`'=13 \gdef \verbatimplus{\catcode`'=13 \chardef '=13 }}
\makeatother

% enables backticks in verbatim
\makeatletter
{\catcode`\`=13
\xdef\@verbatim{\unexpanded\expandafter{\@verbatim}\chardef\noexpand`=18 }
}
\makeatother

%\newcommand{\CX}{\mathcal{X}}
%\newcommand{\PMF}{\mathrm{PMF}}
%\newcommand{\PDF}{\mathrm{PDF}}
%\newcommand{\CDF}{\mathrm{CDF}}
%\newcommand{\N}[2]{\mathcal{N}\left(#1,#2\right)}
%\newcommand{\empavg}[2]{\frac{1}{#1}\sum_{i=1}^{#1}\left[#2\right]}
%\newcommand{\E}[1]{{\rm I\kern-.3em E}\left[#1\right]}
%\newcommand{\Var}[1]{\mathrm{Var}\left[#1\right]}
%\newcommand{\Cov}[1]{\mathrm{Cov}\left[#1\right]}
%\newcommand{\bias}[1]{\operatorname{Bias}\left[#1\right]}
%\newcommand{\sign}[1]{\operatorname{sign}\left(#1\right)}
%\newcommand{\df}[1]{\text{df}\left(#1\right)}
%\def\ci{\perp\!\!\!\perp}
%\newcommand{\argmax}[1]{\underset{#1}{\operatorname{argmax}}}
%\newcommand{\argmin}[1]{\underset{#1}{\operatorname{argmin}}}
%\newcommand{\iid}{\stackrel{\mathrm{iid}}{\sim}}


\usepackage{algpseudocode}
\algtext*{EndWhile}% Remove "end while" text
\algtext*{EndIf}% Remove "end if" text
\algtext*{EndFor}% Remove "end if" text
\algtext*{EndFunction}% Remove "end if" tex

\newcommand{\reals}{\mathbf{R}}
\newcommand{\ints}{\mathbf{Z}}
\newcommand{\rationals}{\mathbf{Q}}

\usepackage{cancel}

\newcommand\SetSymbol[1][]{\nonscript\:#1\vert\allowbreak\nonscript\:\mathopen{}}
\providecommand\given{} % to make it exist
\DeclarePairedDelimiterX\Set[1]\{\}{\renewcommand\given{\SetSymbol[\delimsize]}#1}

\newcommand{\smallcirc}{\mathbin{\text{\raisebox{0.2ex}{\scalebox{0.6}{$\circ$}}}}}
\newcommand*\dif{\mathop{}\!\mathrm{d}}
\newcommand{\pvalue}{\emph{p}-value}

\usepackage{psfrag}
\usepackage{epsfig}

\usepackage{fancyhdr}
\usepackage{float}
\usepackage{pdfpages}

\renewcommand{\floatpagefraction}{1}
\renewcommand{\topfraction}{1}
\renewcommand{\bottomfraction}{1}
\newcommand{\comment}[1]{}
\renewcommand{\a}{\alpha}
\newcommand{\1}[2]{\mathds{1}_{#1 \times #2}}
\newcommand{\one}{\mathbf{1}}
\renewcommand{\b}{\beta}
\newcommand{\e}{\varepsilosn}
\newcommand{\R}{\mathbb{R}}
\renewcommand{\u}{\mathbf{u}}
\newcommand{\rank}{\mbox{rank}}
\renewcommand{\sp}{\mbox{span}}
\newcommand{\prj}{\mbox{proj}}
%\newcommand\esp{{\mathbb{E}}}
\newcommand\prob{{\mathbb{P}\text{r}}}
\newcommand\var{{\mathbb{V}\text{ar}}}
%\newcommand\cov{{\mathbb{C}\text{ov}}}
%\newcommand\corr{{\mathbb{C}\text{orr}}}
\newcommand\loin{{\mathcal N}}
\newcommand\loif{{\mathcal F}}
\newcommand\loix{{{\mathcal X}2}}
\newcommand\loit{{{\mathcal S}\text{t}}}
\newcommand{\E}{\mathbb{E}}
\newcommand{\cov }{\mathrm{cov}}

\newtheorem{lemma}{Lemma}
\newtheorem{theorem}{Theorem}
\newtheorem{proof}{Proof}
\newtheorem{define}{{Definition}}
\newtheorem{algorithm}{Algorithm}

%Interligne (pour la version finale: 1.15 parait un bon compromis)
\renewcommand{\baselinestretch}{1.15}
\newcommand{\TMLE}{\mathrm{TMLE}}
% caption (pris dans /usr/local/lib/tex/inputs/book.sty)
\makeatletter
\long\def\@makecaption#1#2{
   \vskip 11pt
   \setbox\@tempboxa\hbox{\parbox{12cm}{\footnotesize \hspace*{0.5cm}
    {\bf #1:}~~{\sl #2}}}
   \ifdim \wd\@tempboxa >\hsize   % IF longer than one line:
       \unhbox\@tempboxa\par      %   THEN set as ordinary paragraph.
     \else                        %   ELSE  center.
       \hbox to\hsize{\hfil\box\@tempboxa\hfil}
   \fi}
\makeatother
