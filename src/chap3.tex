\chapter{Software}

To support widespread use of this newly developed methodology, an R package,
``biotmle'', which provides a generalized implementation of this biomarker
discovery procedure, has been made publicly available. As previously described
in~\ref{estimation}, the method, based on targeted minimum loss-Based
estimation (TMLE)~\cite{van2011targeted} and a generalization of the moderated
t-statistic of~\cite{smyth2004linear}, is designed for use with both microarray
and next-generation biological sequencing data. The statistical approach made
available in this software package relies on the use of TMLE to rigorously
evaluate the association between a set of potential biomarkers and another
variable of interest while adjusting for potential confounding from another set
of user-specified covariates. The implementation is in the form of a package for
the R language for statistical computing \cite{R}.

There are two principal ways in which the biomarker discovery techniques in
the ``biotmle'' R package may be used: to evaluate the association between (1) a
phenotypic measure (say, environmental exposure) and a biomarker of interest,
and (2) an outcome of interest (e.g., survival status at a given time) and a
biomarker measurement, both while controlling for background covariates (e.g.,
BMI, age). By using a TMLE-based procedure to estimate the Average Treatment
Effect (ATE) in a targeted manner, the package produces interpretable results
in the form of a variable importance measure (VIM) (see~\cite{van2011targeted}
for an extended discussion), making the ``biotmle'' package well-suited for
applications in bioinformatics, genomics, and molecular epidemiology.

While the principal table of results produced by this R package matches those
produced by the immensely popular ``limma'' R package \cite{smyth2005limma},
the package provides several unique utilities, including several plotting
methods --- for example, a heatmap based on the recently developed ``superheat''
R package~\cite{barter2017superheat} --- and a custom ``biotmle'' class,
based on the popular ``SummarizedExperiment'' class. While the R package is
currently publicly available at \url{https://github.com/nhejazi/biotmle},
submission of the software package to the centralized repository maintained by
the Bioconductor project~\cite{gentleman2004bioconductor} is underway.
