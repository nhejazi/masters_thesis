\chapter{Discussion}

The goal of the present paper is the introduction of an automated, robust
method for analyzing high-dimensional exposure and Omics data with relatively
modest sample sizes. In the provided examples, the challenge was two-fold,
including both simultaneous inference for a large number of comparisons and
adjustment for potential confounders, all in the context of a large statistical
model and small numbers of biological replicates. Since the goal here is
estimation within a very large statistical model, the technique must involve
data-adaptive estimation, while still providing trustworthy statistical
inference and estimators grounded in semiparametric efficiency theory. That is,
given the parameter of interest and the nature of the statistical model, we
maintain that the choice guiding the algorithm should not be \textit{ad hoc},
but rather based on the relative efficiency of competing estimators. We have
proposed methods that draw on existing work in statistical genomics and merge
these with modern proposals for the analysis of variable importance, ultimately
yielding a procedure that data-adaptively identifies promising biomarkers from
a large set and that can be applied to data generated from experiments
belonging to a large class of study designs.

We illustrated the method using an example miRNA data set (featuring benzene
exposure) by applying, on a probe-by-probe basis, the outlined approach,
combining TMLE with the moderated t-statistic to estimate the association of
each potential biomarker with exposure. Thus, we present a flexible
generalization of the moderated t-statistic to the case of asymptotically
linear paramaters, obtaining robust small-sample inference, derived from
influence curve-based estimation of the parameter of interest. The results
suggest that instabilities inherent in small-sample inference can be
ameliorated by combining this asymptotically efficient estimator of the ATE
(based on TMLE) with the moderated t-statistic (implemented in {\em Limma}; in
our example, this results in the isolation of fewer statistically significant
biomarkers. Since application of the {\em Limma} framework has no impact on
asymptotics -- the adjustment to the within probe inference becomes negligible
as sample size grows -- we can readily use the asymptotic theory underlying
TMLE.

This combination of existing methods offers many advantages: 1) it estimates
target parameters relevant to specific scientific questions, in the presence of
many confounders, without placing assumptions on the underlying statistical
model; 2) it uses the theoretical optimality of loss-based estimation via the
Super Learner algorithm, which optimally balances the bias-variance tradeoff in
finite samples by appropriately choosing a level of parsimony to match the
information available in the sample; 3) its reliance on TMLE-based estimators
reduces residual bias and adds an appropriate degree of smoothing, making
influence curve-based based inference available for the target parameters of
interest; and 4) it robustifies inference by using the moderated t-statistic to
derive joint inference with fewer false positives than would result from
otherwise poor estimation of the sampling variability of the estimator. The
result is a theoretically sound, data-adaptive estimation procedure, based on
pre-specified, flexible learning algorithms, that guarantees robust statistical
inference. While the continuing development of new biotechnologies promises new
insights into the myriad relationships between biomarkers and health,
procedures like the one presented here will surely be necessary to ameliorate
the pitfalls of increasing dimensionality of the scientific problems of
interest, by providing a rigorous and generalizeable statistical framework for
accurate, robust, and conservative biomarker discovery.
