% (This file is included by thesis.tex; you do not latex it by itself.)

\begin{abstract}

% The text of the abstract goes here.  If you need to use a \section
% command you will need to use \section*, \subsection*, etc. so that
% you don't get any numbering.  You probably won't be using any of
% these commands in the abstract anyway.

The exploratory analysis of high-dimensional biological sequencing data has
received much attention for its ability to allow the simultaneous screening of
numerous biological characteristics. While there has been an increase in the
dimensionality of such data sets in studies of environmental exposure and
biomarkers, two important questions have received less attention than deserved:
(1) how can individual estimates of independent associations be derived in the
context of many competing causes while avoiding model misspecification, and (2)
how can accurate small-sample inference be obtained when data-adaptive
techniques are employed in such contexts. The central focus of this paper is on
variable importance analysis in high-dimensional biological data sets with
modest sample sizes, using semiparametric statistical models. We present a
method that is robust in small samples, but does not rely on arbitrary
parametric assumptions, in the context of studies of gene expression and
environmental exposures. Such analyses are faced not only with issues of
multiple testing, but also the problem of trying to tease out the associations
of biological expression measures with exposure, among confounds such as age,
race, and smoking. Specifically, we propose the use of targeted minimum
loss-based estimation (TMLE), along with a generalization of the moderated
empirical Bayes statistics of Smyth, relying on the influence curve
representation of a statistical target parameter to obtain estimates of variable
importance measures (VIM). The result is a data-adaptive approach that can
estimate individual associations in high-dimensional data, even in the presence
of relatively small sample sizes.

\end{abstract}
